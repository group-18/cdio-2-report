\chapter{Implementering}
I projektet har vi primært brugt brugt følgende software:
\begin{enumerate}
    \item \textbf{Visual Studio Code}
    \\ Vi valgte fra starten at skrive i LaTeX, for at lette skrivningsprocessen og få en homogen rapport.
    Samtidig er filtypen .tex understøttet af Git, til versionsstyring.
    \item \textbf{Tower}
    \\Tower er et af de mange GIT-UI på markedet, og det fungerer eminent, til de funktioner, vi har benyttet os af.
    \item \textbf{IntelliJ}
    \\Vores program er skrevet i IntelliJ, sideløbende med Eclipse for at teste, om slutproduktet fungerer for flere forskellige maskiner og i forskellige programmer.
    \item \textbf{Asana}
    \\Asana er et projektstyringsprogram, hvor man kan skabe opgaver og tildele personer til opgaver med eventuelle deadlines.
    \item \textbf{Slack}
    \\Slack er en \textit{digital workspace}, hvor man kan kommunikere med projekts medarbejdere.
\end{enumerate}
\section{Brugervejledning for programmet}
Programmets brugerinterface er meget simpelt sat op, og guider brugeren nemt igennem hvad han/hun skal gøre.
Når man starter programmet op, vil det åbne en GUI, hvor den vil starte med at spørge om spiller 1's navn.
Dette indtaster brugeren og trykker herefter på ok, herefter gentages denne proces for spiller 2.
Programmet vil herefter printe reglerne styk for styk ud, når brugeren trykker videre.
Når reglerne er blevet printet, vil den skrive, hvis tur det er.
Der vil så komme en knap frem, der hedder kast, hvorved at når der bliver trykket, vil der blive simuleret et terningeslag, og terningerne vises på GUI'en.
Herefter vil programmet opdatere spillerens score, og give turen videre (hvis spilleren ikke slog en sum af 10).
Dette fortsætter programmet med indtil en af spillerne har opnået en pengebeholdning på 3000.
Brugeren kan desuden holde musen over de forskellige felter, for at få beskrivelsen af feltet, og den værdi den påvirker pengebeholdningen med.