\chapter{Implementering}
Vi har i dette projekt tager udgangspunkt i CDIO1, hvor vi desværre ikke brugte særlig mange metoder og generelt set fulgte vi en ringe arbejdsproces.
I det her projekt var det nødvendigt for os at programmere, sideløbende med en analyse- og designfase.
Det viste sig at være en kæmpe hjælp at have et klassediagram, mens man koder, fordi man på klassediagrammet har en idé om, hvordan koden skal se ud.
\\
Det er dog klart, at diagrammerne bliver ændret i løbet af kodningen, fordi det er irrationelt at tro, at uforudsete ting ikke kan forekomme.
I 'Udviklingsmetoder til IT-Systemer' har vi altså lært ikke at følge en vandfaldsmodel, men en agil model, hvor projektet deles op i iterationer, som det vises i kapitlet \textit{Projektplanlægning.}.
\\
Det viste sig umiddelbart at være en rigtig god idé at have inddelt programmet i flere klasser med hver deres ansvar for at opnå 'low coupling' og 'high cohesion', hvilket vi i det senere projekt CDIO3 højst tænkeligt kommer til at være rigtig glade for.

\section{Brugervejledning for programmet}
Programmets brugerinterface er meget simpelt sat op, og guider brugeren nemt igennem hvad han/hun skal gøre.
Når man starter programmet op, vil det åbne en GUI, hvor den vil starte med at spørge om spiller 1's navn.
Dette indtaster brugeren og trykker herefter på ok, herefter gentages denne proces for spiller 2.
Programmet vil herefter printe reglerne styk for styk ud, når brugeren trykker videre.
Når reglerne er blevet printet, vil den skrive, hvis tur det er.
Der vil så komme en knap frem, der hedder kast, hvorved at når der bliver trykket, vil der blive simuleret et terningeslag, og terningerne vises på GUI'en.
Herefter vil programmet opdatere spillerens score, og give turen videre (hvis spilleren ikke slog en sum af 10).
Dette fortsætter programmet med indtil en af spillerne har opnået en pengebeholdning på 3000.
Brugeren kan desuden holde musen over de forskellige felter, for at få beskrivelsen af feltet, og den værdi den påvirker pengebeholdningen med.