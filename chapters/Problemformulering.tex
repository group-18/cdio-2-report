\chapter{Problemformulering}
Er det muligt at producere et brætspil, bestående af to spillere med hver deres pengebeholdning, som finder sted på felter der rækker fra 2 - 12 felter?

\section*{Spillets regler:}
\label{Sec:rules}
Reglerne for dette spil finder i første omgang megen inspiration fra CDIO 1 opgaven.
Der er tale om et brætspil hvor to spillerer på skift kaster med to terninger, hvor så resultatet udgør en sum bestående af en addition mellem terning et og to.
CDIO 2 opgaven introducerer nye funktioner og spilleregler, herunder en pengebeholdning for hver spiller og felter med hver sin unik værdi, der påvirker spillerens pengebeholdning positivt eller negativt.

\vspace{5mm}

\noindent I opstartsfasen får hver spiller tildelt en pengebeholdning på 1000.
I mellemtiden slår to spillerer på skift med to terninger, og lander på felter nummereret fra 2 - 12, med værdier der tager udgangspunkt i opgaveformuleringen.
Spillet sluttes når en af spillerne når en pengebeholdning på 3000.

\vspace{5mm}
\begin{table}[H]
    \begin{center}
        \begin{tabular}{ | c | l | p{7cm}|}
            \hline
            \textbf{Felt} & \textbf{Navn på felt} & \textbf{Værdi} \\ \hline
            2.  &   Tower                   & + 250 \\ \hline
            3.  &   Crater                  & - 100 \\ \hline
            4.  &   Palace gates            & + 100 \\ \hline
            5.  &   Cold Desert             & - 20 \\ \hline
            6.  &   Walled City             & + 180 \\ \hline
            7.  &   Monastery               & 0 \\ \hline
            8.  &   Black cave              & - 70 \\ \hline
            9.  &   Huts in the mountain    & + 60 \\ \hline
            10. &   The Werewall            & - 80 
            (men spilleren får en ekstra tur) \\ \hline
            11. &   The pit                 & - 50 \\ \hline
            12. &   Goldmine                & + 650 \\ \hline
            \hline
        \end{tabular}
    \end{center}
    \caption{Felter i spillet}
    \label{table:fields}
\end{table}