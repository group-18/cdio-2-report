\chapter{Indledning}
I denne rapport vil vi gennemgå en opgave, som spilfirmaet \textit{IOOuterActive} har modtaget, og vi, som udviklere for dette firma, vil skabe dette spil, som opgaven beskriver.
Opgaven vil i denne rapport optræde under navnet 'CDIO2' \footnote{Denne indledning er en revideret udgave af indledningen fra den tidligere CDIO1-rapport af samme gruppe.}.
\\\\CDIO2 er et projekt, der går ud på at skrive et spil, der går ud på at optjene point (Se kapitel \ref{Sec:rules}).
Ligesom et matadorspil indeholder spillet felter, der hver har en unik effekt på den pågældende spillers pointpulje.
Spillet kræver to spillere, og spilleren interagerer med spillet vha. simple knapper (\textit{OK} og \textit{Kast}).
Som en feature er spillet oversat i flere forskellige sprog, for at forbedre brugeroplevelsen (\textit{UX}), og man kan frit skifte til det ønskede sprog, som pr. standard er dansk.
\\I rapporten fremgår også en dybdegående test af koden, både UnitTesting af terningernes fordeling, fordelingen af antal sejre som hhv. spiller 1 og spiller 2 samt en kodedækningstekst.
\\\\Vi har i denne opgave forsøgt at arbejde på en agil måde, og har derved valgt at arbejde med UP (Unified Process). 
UP går i sin helhed ud på, at dele projektet op i iterationer.
Normalt vil disse vare 2-6 uger, men i dette korte projekt har vi ikke tidsbestemte iterationer.
