\section{Dokumentation af kode}

\noindent For at kunne skabe et passende overblik over hvordan og hvorledes koden er konstrureret, har vi valgt at dele dokumentationen op i nogle overordnede kategorier, som reflekterer spillets forskellige aspekter.
Der er tale om klasserne Main, Board, Dice, Game samt en tekstfil, der indeholder oversættelsen.


\subsection{Main, Game og Board}
Main klassens formål er at danne ramme om applikationen.
Dette gøres bedst ved kun at bruge den til at sætte konfigurationer op samt statte de vigtige dele i applikationen.
I vores tilfælde er dette Game klassen hvor den også kører nædvendige metoder herpå for at komme i gang.
\\

\noindent\textbf{Game} \\
Vi har valgt at give et udsnit af Game klassen, nemlig vores \lstinline{play()} metode.
Denne metode står for hele flowet i applikationen og afspejler implementeringen i sekvens diagrammet som ses på figur \ref{fig:sequence_diagram}.

\begin{lstlisting}
public void play() {
    this.printWelcomeMessage();

    Player currentPlayer;
    boolean winnerFound;

    do {
        currentPlayer = getNextPlayer();

        this.print(this.translate.t("turn.currentPlayer", new String[] {currentPlayer.getName()}));

        GUI.getUserButtonPressed("", "Kast");

        do {
            GUI.removeAllCars(currentPlayer.getName());

            this.rollDies();

            GUI.setCar(this.sum(), currentPlayer.getName());
            GUI.setDice(this.dies[0].getFaceValue(), 3, 8, this.dies[1].getFaceValue(), 4, 8);

            int fieldNumber = this.sum() - 2;

            this.print(new String[] {
                    this.translate.t("turn.rollResult", new String[] {"" + sum()}),
                    this.translate.t("turn.position", new String[] {this.fields[fieldNumber][1]}),
                    this.translate.t("field." + this.fields[fieldNumber][0] + ".description"),
            });

            int score = this.fieldValues[fieldNumber];
            currentPlayer.addAmount(score);

            this.print(new String[] {
                this.translate.t("turn.score", new String[] {"" + (score > 0 ? "+" : "") + score}),
                this.translate.t("turn.scoreCurrent", new String[] {"" + currentPlayer.getAmount()}),
            });

            GUI.setBalance(currentPlayer.getName(),currentPlayer.getAmount());
        } while (this.sum() == 10);

        winnerFound = currentPlayer.getAmount() >= 3000;
    } while (! winnerFound);

    this.printScoreBoard(currentPlayer);
}
\end{lstlisting}
\vspace{2ex}

\noindent Metoden er bygget op på to do/while loops som sørger for at hente den næste spiller, kaste terningerne samt udskrive resultatet af kastet og det felt man er landet på.
Det inderste loop (fra linje 14 til 39) eksistere så vi kan give samme spiller en ekstra tur når han/hun rammer felt 10.
Det yderste loop sikrer at hente spilleren og tjekke om han/hun har vundet.
\\

\noindent\textbf{Board}\\

\noindent Spillet gør brug af en tidligere udført GUI til visuel fremvisning af spillerer, brikker samt andre vigtige meddelelser.
I Board klassen bygger vi brættet til denne GUI.
Dette er gjort ved først at importere dets pakker som eksempelvis \lstinline{desktop_resources} og \lstinline{desktop_fields} hvilket tillod os at hente informationer vedrørende strukturer, layout (udseende) m.m. fra den vedlagte GUI.
Eftersom at henvisningerne til de nødvendige ressourcer er etableret, kan vi begynde med at indføre de opkrævede felter.
Dette opnås ved at lave et array ved navn fields bestående af 12 felter.
\lstinline{GUI.create(fields)} metoden tager udgangspunkt i de 12 fields og derefter skaber visuelt de omtalte felter på brættet.
\\


\subsection{Dice, Player og Stash}

\textbf{Dice}\\
\noindent Terningen er i implementeret ved at give brugeren mulighed for selv at vælge antallet af sider.
Dette kan gøres i constructoren.
Når man kaster med terningen gennem metode \lstinline{roll()} vil terningen ud fra antallet af sider sætte en random værdi.
\\

\noindent\textbf{Player}\\
\noindent Player klassen udgør strukturen for spilleren i spillet.
Denne klasse holder styr på spillernavn samt dennes pengebeholdning.
Det er også gennem denne klasse at vi tilføjer og fjerner penge fra spillerens pengebeholdning.
Dette er dog metoder som går direkte videre til Stash klassen.
\\

\noindent\textbf{Stash}\\
Stash klassens formål er at give spillerne en konto hvorpå de kan opbevare deres penge.
Der er her tilknyttet metoder til at indsætte penge i form af metoden \lstinline{addAmount()}.


\subsection{Translate}

For bedst muligt at kunne håndtere oversættelser til flere sprog, er der lavet en Translate klasse.
Denne har til formål at parse en fil med key : value pairs af oversættelser, samt kunne give den rigtige oversættelse.
Vi har valgt at forklare det vigtigste i klassen herefter.
\\

\noindent Dette foregår ved at filen bliver indlæst og parset udfra en struktur hvor oversættelses filen er sat op som \lstinline{key:translation}.
Dette sker i metode \lstinline{parseFile(String filePath)} som kan ses nedenfor.

\begin{lstlisting}
private void parseFile(String filePath)
{
    try {
        InputStream in = this.getClass().getResourceAsStream(filePath);
        BufferedReader reader = new BufferedReader(new InputStreamReader(in));

        String currentLine;
        while ((currentLine = reader.readLine()) != null) {
            if (currentLine.equals("")) {
                continue;
            }

            String[] keyValue = currentLine.split(":", 2);

            String key = keyValue[0];
            String val = keyValue.length == 1 ? "" : this.parseValue(keyValue[1]);

            if (! this.translations.containsKey(key)) {
                this.translations.put(key, val);
            }
        }
    } catch (IOException e) {
        e.printStackTrace();
    }
}
\end{lstlisting}
\vspace{2ex}

\noindent Denne metode indlæser filen fra systemet (linje 3-4) og behandler derefter hver linje som en translation.
Her splittes linjerne på \lstinline{:} og de bliver gemt i et \lstinline{HashMap}.
Vi har valgt at gemme alle oversættelserne i memory da performance for læsning af fil ikke er nær så god som ved læsning fra memory.
\\

\noindent Translate klassen består også af metoden \lstinline{t(String key)} som henter en oversættelse ud fra et givent id.
Denne metode har vi udvidet med en overloaded metode der kan tage et array af strings.
Dette er gjort ved i selve oversættelsen at sætte \lstinline!{{ #0 }}! ind (0 skal være index i det arrayet man sender med).
I metoden benyttes regex (se linje 7 i udsnit fra koden nedenfor) til at finde og replace i oversættelsen før den sendes retur.

\begin{lstlisting}
public String t(String key, String[] variables)
{
    String translation = this.t(key);

    if (! translation.equals(key)) {
        for (int i = 0; i < variables.length; i++) {
            translation = translation.replaceAll("\\{\\{ ?#" + i + " ?\\}\\}", variables[i]);
        }
    }

    return translation;
}
\end{lstlisting}
\vspace{2ex}