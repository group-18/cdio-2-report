\subsection{FURPS+ \& MoSCoW}

\subsubsection{FURPS+}
FURPS+ er en metode til at kategorisere / klassificere krav. Det plus vi ser til sidst i FURPS+ vises på mange forskellige måder, og er ikke altid det samme i alle sammenhænge.

\begin{center}
\begin{tabular}{cll}
       \textbf{F}   &   Functionality   &
       Egenskaber, ydeevne, sikkerhed \\

       \textbf{U}   &   Usability       &
       Menneskelige faktorer, hjælp, dokumentation      \\

       \textbf{R}   &   Reliability     &
       Fejlfrekvens, fejlretning, forudsigelighed     \\

       \textbf{P}   &   Performance     &
       Svartider, nøjagtighed, ydeevne, ressourceforbrug     \\
       \textbf{S}   &   Supportability  &
       Anvendelighed, tilpasningsevne, vedligeholdbarhed     \\
       \textbf{+}   &                   &     \\
       &   Implementation: &   Ressourcebegrænsninger, sprog og værktøjer, hardware   \\

       &   Interface:      &   Begrænsninger forårsaget af kommunikation med eksterne systemer     \\

       &   Operations:     &   Systemstyring i dets operationelle ramme  \\

       &   Packaging:      &   F.eks. en fysisk boks  \\

       &    Legal:         &   F.eks. licenser  \\
\end{tabular}
\end{center}

Med FURPS+ kan vi analysere vores krav, og se om der er     

\subsubsection{MoSCoW}
MoSCoW, som kan bruges til at prioitere krav.
\begin{tabular}{lll}
    \textbf{Mo} &   "Must have"                 & De mest vitale krav, vi ikke kan undgå. \\
    \textbf{S}  &   "Should have"               & Vigtige krav, som ikke er vitale. \\
    \textbf{Co} &   "Could have"                & The 'nice-to-haves' \\
    \textbf{W}  &   "Won’t have (this time)"    & Things that provide little to no value you can give up on
\end{tabular}
\\