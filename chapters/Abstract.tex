\renewcommand{\abstractname}{Abstract}

\begin{abstract}

    \noindent This project is about the making of game between two players, rolling a set of two dice. In comparison to CDIO 1, CDIO 2 introduces a set of new rules and requirements. Our objective is to construct a new game, while meeting the requirements and demands of the customer.\\

    \noindent These requirements and demands translates into the construct of a game between to players with each their own bankroll, fields numbered 2 - 12 with either a positive or negative value to land on, and ends when a bankroll value of 3000 is reached.\\ 
    
    \noindent In CDIO 2 we decided to part the game into different pieces, mainly to avoid having all of our code in one method, and achieve a better structure. Another reason relates to the demand made by the customer, being the possibility to keep the project maintained. The main class contains two methods, which scans for player name and game.play(); which initializes the game (Game.java). Additionally, separate classes for ‘Player’, ‘DiceCup’ and ‘Dice’ are to be found.\\

    \noindent To get a better understanding regarding the vision of the customer, we drew UML-diagrams consisting of use-case diagrams, domain models, system sequence diagrams and last but not least class diagrams. These diagrams was afterwards used as tools in development process of the game, whereas Java was used to write the back- and frontend.\\

    \noindent This paragraph is about results.\\

    \noindent In conclusion we are satisfied with the result, and we believe that the game meets the requirements and demand of the customer.

    \ThisLRCornerWallPaper{0.7}{graphics/dtu/DTU-frise-SH-15}
\end{abstract}