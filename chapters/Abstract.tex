\renewcommand{\abstractname}{Abstract}

\begin{abstract}

    \noindent This project is about the making of a game between two players, rolling a set of two dice. In comparison to CDIO 1, CDIO 2 introduces a set of new rules and requirements. Our objective is to construct a new game, while meeting the requirements and demands from the customer.\\

    \noindent In the game these requirements demand that each player has their own stash of points. It must have fields to land on, numbered from 2 to 12, with individual positive or negative values to land on. When a stash hits the value of 3000 the game ends.\\ 
    
    \noindent In CDIO 2 we decided to part the code of the game game into different pieces. This is mainly to avoid having all of our code in one method, and achieve a better code-structure. Another reason relates to the requirements from the the customer, being the possibility to be able to maintain the project easily. The main class contains two methods, one scans for player name, and game.play(); initializes the game (Game.java). Additionally, separate classes for 'Board', 'Dice', 'Game', ‘Player’, 'Stash', and 'Translate' are to be found in the sourcecode.\\
    
    \noindent To get a better understanding, regarding the vision of the customer, we drew UML-diagrams. Which contains use-case diagrams, domain models, system sequence diagrams and class diagrams. These diagrams was afterwards used, as tools in development process of the game. We have used Java as a back- and frontend to develop the program.\\

    \noindent In the end, we found the results to be satisfying. We succesfully created a program that meets the requirements of the customer. We also succeded in implementing the provided GUI into the game. We created a bunch of test cases to test the different aspects of the program. \\

    \noindent In conclusion we are satisfied with the result, and we believe, that the game meets the requirements and demand of the customer.

    \ThisLRCornerWallPaper{0.7}{graphics/dtu/DTU-frise-SH-15}
\end{abstract}