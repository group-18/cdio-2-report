\chapter{Konfiguration}

Konfigurationen i denne opgave tager udgangspunkt i kundens vision. Kundens vision sætter typisk retningslinjerne til hvordan et projekt skal gribes an, og hvordan og hvorledes de forskellige funktioner skal implementeres.

\section{Minimumskrav}

Kundens vision i CDIO 2 kommer ikke rigtig med nogen form for fyldestgørende beskrivelse af de såkaldte systemkrav der skal implementeres i projektet. 
Der oplyses at spillet skal kunne køre på maskinerne i DTU’s databarer, hvilket i sig selv røber flere ting.
For det første ved vi i forvejen at programmeringssproget er JAVA. 
Vi ved også, at på baggrund af at spillet skal kunne køres samt compiles og afvikles på DTU's maskiner, så er det oplagt at spillet udarbejdes i en JAVA understøttende IDE såsom Eclipse eller eventuelt kunne eksporteres i et format som Eclipse understøtter i tilfælde af at en anden IDE tages i brug. 
Derudover lægges der vægt på at spillet skal kunne køres uden bemærkelsesværdige forsinkelser, hvilket antyder at de implementerede funktioner ikke skal indeholde metoder der er komplicerede mere end nødvendigt. 
Dvs. at indlæsning af ekstra ressourcer i form af udvidelser eller ekstraordinære funktioner skal helst undgås.
\\

\noindent \textbf{Liste over krav}
\begin{itemize}
    \item[--] \textbf{IDE:} Eclipse
    \item[--] \textbf{Programmeringssprog:} JAVA
    \item[--] \textbf{Funktioner:} Mindre sofistikerede metoder, eller evt. opdeling af kode = større effektivitet, ekstra ressourcer i form af udvidelser eller ekstraordinære funktioner
\end{itemize}

\newpage

\section{Vejledning}

Følgende procedure er målrettet et Eclipse miljø, taget i betragtning at projektet skal kunne iscenesættes i Eclipse. Samtidigt skal projektet importeres fra et GIT repository, hvor så en vejledning for dette følger i dette afsnit.\\

\noindent \textbf{Hente projekt fra repository, Git}\\
Det er flere måder at hente et projekt fra et Git repository, hvor det i sidste ende afhænger af om man gør brug af en lokal installation eller web udgave af det omtalte Git klient. I vores tilfælde havde vi gjort brug af Github til håndtering af vores Git repository. Ved lokalisering af projektet på Github skal man blot klikke på ‘Clone or download’, herefter hentes en lokal kopi af det omtalte projekt. For at kunne åbne projektet i eksempelvis Eclipse, skal projektet blot importeres.\\

\noindent \textbf{Compiling, installering og afvikling af kildekoden, Eclipse}\\
Eftersom at projektet er tilgængelig på maskinen lokalt, skal man blot importerer projektet i Eclipse. Dette gøres ved at åbne Eclipse og lokaliserer menuen ‘File’ på menubjælken. Under ‘File’ klikkes derefter på ‘Import’, og typen ‘Projects from Folder or Archive’ vælges. Mappen der indeholder vores lokale Git kopi lokaliseres og sættes som kilde, og processen afsluttes vha. knappen ‘Finish’.\\

\noindent \textbf{\textit{Processen ser som følger:}}\\
\textit{File -> Import -> Projects from Folder or Archive -> Source: Git kopi -> Finish}\\

\noindent For at kunne compile og afvikle programmet, skal man blot klikke på det ikon der repræsentere et trekant omringet af en grøn cirkel. Eventuelt kan man nøjes med at klikke lokaliserer menuen ‘Run’ på menubjælken.
